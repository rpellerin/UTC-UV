\documentclass[11pt,a4paper,oneside,french,svgnames]{report}
\usepackage[utf8]{inputenc} % force the use of utf8
\usepackage[T1]{fontenc} % font encoding, allows accents
\usepackage[top= 1.5cm,bottom=0.5cm, inner=0.5cm, outer=0.5cm]{geometry} % page formatting
\usepackage[francais]{babel} % translate everything in the desired language: table of contents, etc. 'english' can be replaced with 'francais'
\usepackage{parskip} % disable indent
\usepackage{graphicx} % images management
\usepackage{wrapfig} % floating images
\usepackage{array} % allow arrays
\usepackage{fancyhdr} % headers/footers management (overrides empty, plain and headings)
\usepackage{listings} % code insertion (MUST BE WRITTEN AFTER BABEL)
\usepackage{enumitem} % for /setlist
\usepackage{color,soul} % add some colors and hightlight
\usepackage{xcolor} % more colors
\usepackage[hyphens]{url} % auto break lines in URL
\usepackage[hidelinks,  colorlinks  = true, % no borders, colors enabled
                        anchorcolor = blue,
                        linkcolor   = black, % links in table of contents
                        urlcolor    = blue,
                        citecolor   = blue]{hyperref}

\sethlcolor{cyan} % package soul

%%%%%%%%%%%%%%%%%%%%%%%%%% LISTINGS %%%%%%%%%%%%%%%%%%%%%%%%%%
\definecolor{comment}{rgb}{0.12, 0.38, 0.18 } % adjusted, in Eclipse: {0.25, 0.42, 0.30 } = #3F6A4D
\definecolor{keyword}{rgb}{0.37, 0.08, 0.25}  % #5F1441
\definecolor{string}{rgb}{0.06, 0.10, 0.98} % #101AF9

\lstset{
  columns=flexible, %prevent extra spaces
  rulecolor=\color{black!50},
  backgroundcolor = \color{blue!10},
  numbers=none, % line numbering
  showspaces=false,
  showtabs=false,
  tabsize=4,
  breaklines=true,
  showstringspaces=false,
  breakatwhitespace=false,
  commentstyle=\color{comment},
  keywordstyle=\color{keyword},
  stringstyle=\color{string},
  basicstyle=\ttfamily,
  extendedchars=true,
  emph=[2]{In},
  emphstyle=[2]\color{black!70},
  morecomment=[l][\color{blue}]{Out},
  frame=single,
  frameround=tttt,
  framerule=0.3pt,
  framesep=4pt,
  belowcaptionskip=2.1pt,
  literate={à}{{\`a }}1 {â}{{\^a}}1 %                         letter a
           {À}{{\`A}}1 {Â}{{\^A}}1 %                         letter A
           {ç}{{\c{c}}}1 %                                   letter c
           {Ç}{{\c{C}}}1 %                                   letter C
           {é}{{\'e}}1 {è}{{\`e}}1 {ê}{{\^e}}1 {ë}{{\"e}}1 % letter e
           {É}{{\'E}}1 {È}{{\`E}}1 {Ê}{{\^E}}1 {Ë}{{\"E}}1 % letter E
           {î}{{\^i}}1 {ï}{{\"i}}1 %                         letter i
           {Î}{{\^I}}1 {Ï}{{\"I}}1 %                         letter I
           {ô}{{\^o}}1 %                                     letter o
           {Ô}{{\^O}}1 %                                     letter O
           {œ}{{\oe}}1 %                                     letter oe
           {Œ}{{\OE}}1 %                                     letter OE
           {ù}{{\`u}}1 {û}{{\^u}}1 {ü}{{\"u}}1 %             letter u
           {Ù}{{\`U}}1 {Û}{{\^U}}1 {Ü}{{\"U}}1 %             letter U
  % above is a hack to force UTF8 compatibility (only for french)
}

% Usage : \code{main.cpp}{C++}
\newcommand{\code}[2]{\lstset{
  language=#2,
  title={{\setlength{\fboxsep}{1pt}\fcolorbox{orange}{yellow!20}{\sffamily\scriptsize
              \textcolor{gray!10}{\_}{#1}\textcolor{gray!10}{\_}}}}
  }
}

\title{SR02}
\author{Romain PELLERIN}
\date\today

\makeatletter
\let\thetitle\@title
\makeatother

%\parindent=20pt
\fancypagestyle{plain}{
    % Headers
    \renewcommand{\headrulewidth}{0pt}
    \addtolength{\headsep}{-0.5cm}
    \fancyhead[L]{\color{gray}\thetitle}
    \fancyhead[R]{\color{gray}Dernière MaJ : \today}

    % Footers
    \renewcommand{\footrulewidth}{0pt}
    \fancyfoot{}
}

\setlist[itemize,2]{label={$\bullet$}} % use bullets for nested itemize

\pagestyle{plain} % uses fancy

\begin{document}
\scriptsize

\section*{Interblocage de processus}

Conditions nécessaires (simultanées) d'interblocage:
\begin{itemize}
\item Exclusion mutuelle
\item Occupation et attente : Processus occupant au moins une ressource et qui attends d'acquérir des ressources supplémentaires détenues par d'autres processus
\item Pas de réquisition : Les ressources déjà détenues ne peuvent être retirées de force à un processus
\item Attente circulaire
\end{itemize}

\textbf{Si chaque type de ressource possède exactement un seul exemplaire alors: il y a situation d'interblocage SSI le graphe d'allocation possède un circuit}\\
Cas des ressources possédant plusieurs exemplaires : Si le graphe d'allocation est sans circuit alors un processus n'est pas dans une situation d'interblocage (réciproque fausse)\\
=> Interblocage s'il existe un sous-ensemble D (ensemble des sommets formant un cicuit) non vide de processus tel que, pour tout Pi appartenant à D, l'inégalité : DEMANDE[i] <= N - somme(ALLOCATION) avec N[i] nombre max de la ressource i est fausse

Evitement des interblocages => Algorithme du banquier (évalue le risque d'interblocage pouvant être provoqué par une demande de ressource) :
\begin{enumerate}
  \item Vérifier la cohérence de la requête : DEMANDE <= BESOIN
  \item Vérifier la disponibilité des ressources: DEMANDE <= DISPONIBILITE
  \item Accepter temporairement la demande et vérifier l'état du système :\\
DISPONIBLE = BESOIN[i] - DEMANDE\\
ALLOCATION[i] = :old.ALLOCATION[i]+DEMANDE\\
BESOIN[i] = :old.BESOIN[i]+DEMANDE
  \item Appliquer l'algorithme de détermination d'un état sain: TRAVAIL = DISPONIBLE\\Pour tous les Pi, vérifier que BESOIN[i] <= TRAVAIL, si oui TRAVAIL += ALLOCATION[i], si non => système malsain
\end{enumerate}

Détection ds interblocages et reprise : Graphe d'attente si toutes les les ressources possèdent une seule instance (on élimine les noeuds de type ressource)\\
\textbf{Interblocage SSI le graphe d'attente contient un circuit}


 

\section*{Ordonnancement de processus}
Critères d'ordonnancement:
\begin{itemize}
  \item Rendement d'utilisation du CPU : pourcentage de temps pendant lequel le CP est actif => Plus élevé, mieux c'est
  \item Utilisation globale des ressources : assurer une occupation maximale des ressources de la machine et minimiser le temps d'attente pour l'allocation d'une ressource à un processeur
  \item Equité : capacité de l'ordonnanceur (module du système d'exploitation qui attribue le contrôle du CPU à tour de rôle aux différents processus en compétition) à allouer le CPU d'une façon équitable à tous les processus de même priorité (éviter la  famine)
  \item Temps de rotation : durée moyenne nécessaire pour qu'un processus termine son execution
  \item Temps d'attente : durée moyenne qu'un processus passe à attendre le CPU
  \item Temps de réponse : temps moyen qui s'écoule entre le moment oùun utilisateur soumet une requête et celui où il commence à recevoir les réponses
  TMT = (somme(date de fin d'exécutionn - date d'arrivée du processus))/nombre de processus
\end{itemize}

Algorithmes d'ordonnancement:
\begin{itemize}
\item \textbf{FCFS} (First Come First Served) : Les tâches sont ordonnancées dans l'ordre où elles sont venues
\item \textbf{SJF} (Shortest Job First) : Le CPU est attribué au processus dont le temps d'exéction estimé est minimal
\item \textbf{SRT} (Shortest Time Remaining) : Le CPU est attribué au processus qui a le plus petit temps d'execution restant (réévaluation à chaque quantum)
\item \textbf{RR} (Tourniquet) : Le controle du CPu est attribué a chaque processus pendant un quantum
\end{itemize}

Algorithmes avec priorité :  CPU attribué au processus de plus haute priorité


\section*{Threads}

\code{Toutes les fonctions liées aux threads}{c}
\begin{lstlisting}
#include <pthread.h>
pthread_t threads[nth];
pthread_mutex_t mutex;
pthread_cond_t event;
is = pthread_mutex_init(&mutex, NULL);
is = pthread_cond_init (&event, NULL);
is = pthread_mutex_lock(&mutex);
is = pthread_mutex_unlock(&mutex);
is = pthread_cond_signal(&event);
is = pthread_cond_broadcast(&event);
is = pthread_cond_wait(&event,&mutex);
is = pthread_create(&threads[i], NULL, th_fonc, (void *)i);
is = pthread_join( threads[j], &val); 
\end{lstlisting}

\code{Exemple de programme}{c}
\begin{lstlisting}
#include <pthread.h>
#include <stdio.h>
#include <stdlib.h>

#define nth 3 /* nbre de threads a lancer */
#define ifer(is,mess) if (is==-1) perror(mess)

pthread_t threads[nth]; // tableau qui contient les threads
pthread_mutex_t mutex; // MUTEX COMMUN A TOUS LES THREADS
pthread_cond_t event;  // CONDITION DE MUTEX

/* routine exécutée dans les threads */
void *th_fonc (void *arg) {
    int is, numero = (int)arg; // numéro = numéro du thread (i de pthread_create)
    printf("ici thread %d, i=%d\n",numero);
    is = pthread_mutex_lock(&mutex); // retourne 0 si succès. Alternative, non blocante : pthread_mutex_trylock()
    // zone critique
    is = pthread_mutex_unlock(&mutex);   
    return ((void *)(numero)+101); // ou pthread_exit(&numero)
}

int main() {
  int is; // code de retour pour chaque thread
  int i=0; // numéro des threads
  void *val=0; // contient la valeur de retour des threads
  is = pthread_mutex_init(&mutex, NULL); // initialise le mutex
  /* créer les threads */
  for(i=0; i<nth; i++) {
      printf("ici main, création thread %d\n",i);
      is = pthread_create( &threads[i], NULL, th_fonc, (void *)i );
      ifer (is,"err. création thread");
  }
  /* attendre fin des threads */
  for(i=0; i<nth; i++) {
      is = pthread_join( threads[i], &val); // val contient la valeur retournée par chaque thread
      ifer (is,"err. join thread");
      printf("ici main, fin thread j=%d val=%d\n",i,(int)val); // val est de type (void*)
  }
  exit(EXIT_SUCCESS);
}
\end{lstlisting}

\section*{Gestion de la mémoire}
Partitions  multiplec contiguës:
\begin{itemize}
\item Contiguës fixes:
\begin{itemize}
\item files d'attente separées
\item files d'attente communes
\end{itemize}
\item Contiguës dynamique:
\begin{itemize}
\item First fit (selon les adresses croissantes)
\item Best fit (selon les tailles croissantes)
\item Worst fit (selon les tailles décroissantes)
\end{itemize}
\item Contiguë siamoise
\end{itemize}

Algorithme de remplacement de pages:
\begin{itemize}
\item Algorithme optimal : Lors d'un défaut de page, choisit comme victime une page qui ne fera l'objet d'aucune référence ultérieure  ou qui fera l'objet de la référence la plus tardive
\item FIFO : Choisit comme victime la page la plus anciennement chargée
\item LRU  : Choisit comme victime la page ayant fait l'objet de la référence la plus ancienne
\end{itemize}

Calcul du nombre de mots par page :
\begin{equation}
Nombres~de~mots~par~page = nbmots / nbpages
\end{equation}
\end{document}

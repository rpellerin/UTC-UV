\chapter{Conclusion}

À travers l'intégralité de ces différents TD, nous avons pu découvrir les méthodes liées à l'indexation et la recherche de l'information étape par étape. D'abord par la structuration d'information brut en un corpus XML puis la création de tables inverses de lemmes et d'une stop list ; ensuite par un travail de réflexion et d'analyse sur la correction orthographique d'une phrase entrée par un utilisateur ; et enfin par l'analyse lexicale et syntaxique d'une requête en langage naturel à l'aide d'une grammaire SQL.

\medskip

Ces différentes étapes furent ensuite amenées à être rassemblées en une seule application, qui sert d'interface entre l'utilisateur demandant une requête et la base de données questionnée.

\medskip

L'objectif idéal pour ce projet final serait d'obtenir une application à même de traiter presque n'importe quelle requête, quelle que soit la formulation de cette dernière ou le nombre de critères demandés, en prenant en considération les fautes de frappe (et pas seulement les inversions de lettres), etc.

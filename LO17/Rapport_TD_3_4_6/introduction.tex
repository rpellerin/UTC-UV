\chapter{Introduction}

Ce rapport est une synthèse critique d'un ensemble de trois TD réalisés dans le cadre de l'UV LO17. Le but final de ces trois TD est de développer un logiciel permettant à l'utilisateur de saisir une requête en langage naturel afin d'obtenir des résultats extraits d'une base de données.

\medskip

Cette base de données a été préalablement remplie après avoir analysé un certain nombre de pages d'articles d'un site d'information. Ces pages ont été \textit{parsées} afin d'en obtenir des termes significatifs. À présent, nous avons à disposition dans cette base de données plusieurs tables qui contiennent diverses informations organisées, telles que des dates de parution d'articles, les rubriques, les mots significatifs retenus pour chaque article, etc.

\medskip

Notre travail pour ce projet, encore inachevé, a été séparé en trois étapes distinctes :

\begin{enumerate}
  \item \textbf{TD 3} : transformer une phrase en une suite des lemmes. Il s'agissait tout d'abord de permettre à l'utilisateur de rentrer une phrase. Suite à cela, nous devions retirer de cette phrase des mots faisant partie d'une stop list. Puis, finalement, il nous fallait ``transformer'' chaque mot en un ``lemme'', soit directement à partir d'une liste déjà fournie, soit indirectement en utilisant un algorithme (au choix celui de proximité ou de Levenshtein). Il nous a également fallu prendre en considération les erreurs de saisies éventuelles de l'utilisateur.
  \item \textbf{TD 4} : faire de l'analyse syntaxique à l'aide d'Antlr. À partir de la phrase ``\textit{lémmatisée}'' obtenue suite au TD 3, il nous a fallu écrire une grammaire capable de reconnaître chaque terme afin de construire un arbre syntaxique (donc une requête SQL syntaxiquement correcte). Le TD consistait tout d'abord à prendre en main le logiciel à l'aide d'une grammaire déjà fournie puis ensuite à créer notre propre grammaire en partant d'un exemple.
  \item \textbf{TD 6} : ce dernier TP a été l'occasion de créer une petite application Java permettant d'exécuter une requête SQL sur une base de données PostgreSQL mise à disposition sur les serveurs de l'UTC. Il s'agissait principalement d'être capable de transmettre une requête SQL entrée par l'utilisateur et d'afficher les résultats correctement, en utilisant chaque colonne de chaque ligne des résultats.
\end{enumerate}

On peut aisément distinguer ici les différentes étapes qui, une fois les TD ``assemblés'', permettront d'obtenir des résultats SQL à partir d'une requête donnée en langage naturel.

\medskip

Nous présenterons donc dans ce rapport la réalisation de ces trois TP ainsi que la mise en commun de tout le code afin d'obtenir l'application fonctionnelle décrite plus haut.

\documentclass[11pt,a4paper,oneside,french,svgnames]{report}
\usepackage[utf8]{inputenc} % force the use of utf8
\usepackage[T1]{fontenc} % font encoding, allows accents
\usepackage[top= 1.3cm,bottom=0.5cm, inner=0.5cm, outer=0.5cm]{geometry} % page formatting
\usepackage[francais]{babel} % translate everything in the desired language: table of contents, etc. 'english' can be replaced with 'francais'
\usepackage{parskip} % disable indent
\usepackage{graphicx} % images management
\usepackage{wrapfig} % floating images
\usepackage{array} % allow arrays
\usepackage{fancyhdr} % headers/footers management (overrides empty, plain and headings)
\usepackage{listings} % code insertion (MUST BE WRITTEN AFTER BABEL)
\usepackage{enumitem} % for /setlist
\usepackage{color,soul} % add some colors and hightlight
\usepackage{xcolor} % more colors
\usepackage[hyphens]{url} % auto break lines in URL
\usepackage[hidelinks,  colorlinks  = true, % no borders, colors enabled
                        anchorcolor = blue,
                        linkcolor   = black, % links in table of contents
                        urlcolor    = blue,
                        citecolor   = blue]{hyperref}

\sethlcolor{cyan} % package soul

%%%%%%%%%%%%%%%%%%%%%%%%%% LISTINGS %%%%%%%%%%%%%%%%%%%%%%%%%%
\definecolor{comment}{rgb}{0.12, 0.38, 0.18 } % adjusted, in Eclipse: {0.25, 0.42, 0.30 } = #3F6A4D
\definecolor{keyword}{rgb}{0.37, 0.08, 0.25}  % #5F1441
\definecolor{string}{rgb}{0.06, 0.10, 0.98} % #101AF9

\lstset{
  columns=flexible, %prevent extra spaces
  rulecolor=\color{black!50},
  backgroundcolor = \color{blue!10},
  numbers=none, % line numbering
  showspaces=false,
  showtabs=false,
  tabsize=4,
  breaklines=true,
  showstringspaces=false,
  breakatwhitespace=false,
  commentstyle=\color{comment},
  keywordstyle=\color{keyword},
  stringstyle=\color{string},
  basicstyle=\ttfamily,
  extendedchars=true,
  emph=[2]{In},
  emphstyle=[2]\color{black!70},
  morecomment=[l][\color{blue}]{Out},
  frame=single,
  frameround=tttt,
  framerule=0.3pt,
  framesep=4pt,
  belowcaptionskip=2.1pt,
  literate={à}{{\`a }}1 {â}{{\^a}}1 %                         letter a
           {À}{{\`A}}1 {Â}{{\^A}}1 %                         letter A
           {ç}{{\c{c}}}1 %                                   letter c
           {Ç}{{\c{C}}}1 %                                   letter C
           {é}{{\'e}}1 {è}{{\`e}}1 {ê}{{\^e}}1 {ë}{{\"e}}1 % letter e
           {É}{{\'E}}1 {È}{{\`E}}1 {Ê}{{\^E}}1 {Ë}{{\"E}}1 % letter E
           {î}{{\^i}}1 {ï}{{\"i}}1 %                         letter i
           {Î}{{\^I}}1 {Ï}{{\"I}}1 %                         letter I
           {ô}{{\^o}}1 %                                     letter o
           {Ô}{{\^O}}1 %                                     letter O
           {œ}{{\oe}}1 %                                     letter oe
           {Œ}{{\OE}}1 %                                     letter OE
           {ù}{{\`u}}1 {û}{{\^u}}1 {ü}{{\"u}}1 %             letter u
           {Ù}{{\`U}}1 {Û}{{\^U}}1 {Ü}{{\"U}}1 %             letter U
  % above is a hack to force UTF8 compatibility (only for french)
}

% Usage : \code{main.cpp}{C++}
\newcommand{\code}[2]{\lstset{
  language=#2,
  title={{\setlength{\fboxsep}{1pt}\fcolorbox{orange}{yellow!20}{\sffamily\scriptsize
              \textcolor{gray!10}{\_}{#1}\textcolor{gray!10}{\_}}}}
  }
}

\title{NF17 - Oracle}
\author{Romain PELLERIN}
\date\today

\makeatletter
\let\thetitle\@title
\makeatother

%\parindent=20pt
\fancypagestyle{plain}{
    % Headers
    \renewcommand{\headrulewidth}{0pt}
    \addtolength{\headsep}{-0.5cm}
    \fancyhead[L]{\color{gray}\thetitle}
    \fancyhead[R]{\color{gray}Dernière MaJ : \today}

    % Footers
    %\renewcommand{\footrulewidth}{0.1pt}
    %\fancyfoot[C]{Université de Technologie de Compiègne}
    %\fancyfoot[LE]{\ifnum\thepage>0 \thepage \fi}
    %\fancyfoot[RO]{\ifnum\thepage>0 \thepage \fi}
}

\setlist[itemize,2]{label={$\bullet$}} % use bullets for nested itemize

\pagestyle{plain} % uses fancy

\begin{document}
\scriptsize

\begin{itemize}
  \item Pas de type \lstinline{BOOLEAN}
  \item Pas de mode \lstinline{AUTOCOMMIT}, il faut utiliser le mot-clé \lstinline{COMMIT;}
  \item \textbf{Importer un CSV sous SQL Developer} : clic droit sur le dossier tables à gauche dans SQL Developer et choisir "Import data"
\end{itemize}

\subsection*{SQL2}
\code{PL/SQL}{SQL}
\begin{lstlisting}
SELECT table_name FROM dba_tables;
SELECT table_name FROM all_tables; -- Même résultat
SELECT table_name FROM user_tables; -- Que mes tables;
SELECT * FROM ALL_OBJECTS WHERE OBJECT_TYPE IN ('FUNCTION','PROCEDURE','PACKAGE');

DROP TABLE USER CASCADE CONSTRAINTS;
CREATE TABLE USER ( 
  prenom varchar2(20), 
  nom varchar2(20), 
  specialite varchar2(20), 
  dateInscription TIMESTAMP DEFAULT CURRENT_TIMESTAMP NOT NULL,
  sexe char check (sexe in ('m','f')),
  constraint PK_MEMBRE primary key (prenom),
  constraint FK_MEMBRE_SPECIALITE foreign key (specialite) references SPECIALITE(intitule)
);

create sequence numerotation_projet -- Voir la cheatsheet sur PostgreSQL pour plus d'info
INCREMENT BY 1 START WITH 1 NOMAXVALUE NOCYCLE CACHE 10; -- Cache améliore les performances d'accès à la mémoire

INSERT INTO Projet VALUES (numerotation_projet.NEXTVAL, 'Dupont', TO_DATE('06052015', 'DDMMYYYY'), TO_DATE('07052015', 'DDMMYYYY'), 'Pierre', SYSDATE); -- On peut aussi utiliser 'numerotation_projet.CURRVAL', SYSTIMESTAMP contient l'heure, en plus de SYSDATE
COMMIT;
--------------------------------------------------------------FONCTION--------------------------------------------------------------
CREATE OR REPLACE FUNCTION fDemande (num_produit in number) RETURN varchar
IS
qte_vendue NUMBER;
begin
  select sum(qte) INTO qte_vendue from ligne_fact lf where lf.produit = num_produit;
  IF (qte_vendue > 15) THEN
    RETURN ('forte');
  ELSIF (qte_vendue between 11 and 15) THEN
    RETURN ('moyenne');
  ELSE
    RETURN ('faible');
  END IF;
END fDemande;
/
SHOW ERRORS;
--------------------------------------------------------------PROCEDURE--------------------------------------------------------------
CREATE OR REPLACE PROCEDURE nb_fact(num_client IN NUMBER, nb OUT NUMBER, ca OUT NUMBER)
AS
BEGIN
 dbms_output.put_line('Hello World!');
 
 SELECT count(f.num) into nb FROM facture f where f.client = num_client;
 
 SELECT SUM(P.PRIX*LF.QTE) into ca FROM FACTURE F, PRODUIT P, LIGNE_FAC LF WHERE F.NUM = LF.FAC AND P.NUM = LF.prod AND F.cli = n_cli;
END nb_fact;
/
------------------------------------------------------------BLOC ANONYME------------------------------------------------------------
SET SERVEROUTPUT ON
DECLARE
  CURSOR c_client IS SELECT num FROM client;
  nb NUMBER; 
  ca NUMBER;
BEGIN
  For client IN c_client Loop -- ligne de requete qui contient que 'num'
    nb_fact(client.num,nb,ca);
    dbms_output.put_line('the result for client ' || client.num || ' is ' || NB || ' (nb) and ' || CA || ' (ca)'); -- concaténation
  End loop;
END;
/
---------------------------------------------------------------TRIGGER---------------------------------------------------------------
create or replace TRIGGER STOCK_MAJ
AFTER INSERT ON ligne_fact FOR EACH ROW
DECLARE
  NEW_STOCK NUMBER;
BEGIN
  UPDATE PRODUIT SET PRODUIT.STOCK = PRODUIT.STOCK - :new.qte WHERE PRODUIT.num= :new.produit;
  SELECT stock INTO NEW_STOCK from produit where num = :new.produit;
  if (new_stock < 5) THEN
    INSERT INTO JOURNALISATION VALUES ('Attention : rupture de stock imminente',sysdate,:new.produit, NEW_STOCK);
  END IF;
END;
/
----------------------------------------------Afficher le contenu de toutes les tables----------------------------------------------
SET SERVEROUTPUT ON
DECLARE
BEGIN
  FOR t IN (SELECT table_name FROM all_tables WHERE owner <> 'SYS' and data_type LIKE '%CHAR%') LOOP
    dbms_output.put_line(SELECT * FROM t.table_name);
  END LOOP;

END;
/
\end{lstlisting}

\newpage
\subsection*{SQL3 - Relationnel objet}
\code{PL/SQL}{SQL}
\begin{lstlisting}
CREATE OR REPLACE TYPE Client AS OBJECT (
  num INTEGER,
  nom varchar2(30),
  prenom varchar2(30),
  adresse varchar2(50),
  date_nais date,
  tel varchar2(30),
  sexe char
);
/
--------------------------------------------------------------Héritage--------------------------------------------------------------
CREATE TYPE sous-type UNDER sur_type ( déclarations ou surcharges );
------------------------------------------------------------------------------------------------------------------------------------
CREATE TABLE tClient OF Client (
  primary key (num),
  check (sexe in ('m', 'f'))
);
/
CREATE OR REPLACE TYPE LigneFacture AS OBJECT (
  qte integer,
  prod REF Produit, -- prod est un OID d'un objet Produit
  prod2 Produit -- prod EST l'objet
);
/
CREATE OR REPLACE TYPE RefLigneFacture AS OBJECT (
  ligneFacture REF LigneFacture
);
/
CREATE OR REPLACE TYPE ListeRefLigneFacture AS TABLE OF RefLigneFacture; -- Une table de REFERENCES vers des objets LigneFacture
CREATE OR REPLACE TYPE ListeLigneFacture AS TABLE OF LigneFacture; -- Une table d'objets
/
CREATE OR REPLACE TYPE Facture AS OBJECT (
  num integer,
  date_etabli date,
  fk_client REF Client,
  ligness ListeLigneFacture, -- Contient une collection de LigneFacture
  member function total return integer, -- Déclaration d'une fonction, implémentée plus bas
  member function quantite return integer
);
/
CREATE TABLE tFacture OF Facture (
  PRIMARY KEY (num),
  SCOPE FOR (fk_client) is tClient -- tClient est le nom de la table que l'on vise
)
NESTED TABLE lignes STORE AS table_lignes; -- On pourra faire Facture.lignes.COLONNE

CREATE OR REPLACE TYPE BODY Facture IS
  MEMBER FUNCTION total RETURN integer
  IS
    vtotal integer;
  BEGIN
    select sum(lf.qte * lf.prod.prix) INTO vtotal FROM tFacture F, table(f.ligness) lf WHERE f.num = self.num;
    RETURN vtotal;
  END total;
  MEMBER FUNCTION quantite RETURN integer
  IS
    vquantite integer;
  BEGIN
    select sum(lf.qte) INTO vquantite FROM tFacture F, table(f.ligness) lf WHERE f.num = self.num;
    RETURN vquantite;
  END quantite;
END;
/
-------------------------------------------------------------------------------------------------------------------------------------
DECLARE
ref_client_1 ref CLIENT;
ref_client_2 ref CLIENT;
ref_produit_1 ref PRODUIT;
ref_produit_2 ref PRODUIT;
ref_produit_3 ref PRODUIT;
BEGIN
SELECT REF(c) into ref_client_1
from tClient c
WHERE c.num = 1;
-- ...
INSERT INTO tFacture (num, date_etabli, clientt, ligness)
VALUES (
  6, SYSDATE, ref_client_1, ListeLigneFacture(
    LigneFacture(3, ref_produit_1),
    LigneFacture(2, ref_produit_2)
  )
);
END;
/
\end{lstlisting}
\section*{Niveau logique}
\subsection*{Objet et référence d'objet}
\textbf{Un OID ne peut pas être une FK, utiliser string par exemple}\\\\
UN OBJET CONTIENT UN OBJET\\
Type typBureau : <nom: string, numero: integer>\\
Type typIntervenant : <nom: string, bureau: typBureau> -- Composition entre bureau et intervenant, CONTIENT L'OBJET. Il ne faudra pas créer de table pour Bureau

UN OBJET REFERENCE UN OBJET\\
Bureau de typBureau(\#nom) -- Table\\
Type typIntervenant2 : <nom: string, bureau =>o Bureau> -- Bureau est une TABLE et non un TYPE, ici l'objet EST REFÉRENCÉ
\subsection*{Tables imbriquées (collections)}
Type RefBureau : <bureau =>o Bureau>\\
Type ListRefBureau: collection de <RefBureau> -- Collection de références

Type ListBureau: collection de <typBureau> -- Collection d'objets
\end{document}
